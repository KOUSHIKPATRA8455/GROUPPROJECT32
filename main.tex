\documentclass[a4paper,12pt]{article}
\usepackage{amsmath}
\usepackage{listings}
\usepackage{xcolor}
\usepackage{geometry}
\geometry{left=1in, right=1in, top=1in, bottom=1in}
\usepackage{tikz}
\usetikzlibrary{calc}
\lstset{
  language=C,                              % Set language to C
  backgroundcolor=\color{lightgray},       % Set the background color
  frame=single,                            % Add a single frame around the listing
  xleftmargin=1.5em,                       % Left margin to ensure code stays inside
  xrightmargin=1.5em,                      % Right margin for same reason
  numbers=left,                            % Line numbers on the left
  numberstyle=\tiny\color{black},          % Line number style
  stepnumber=1,                            % Show line numbers for every line
  numbersep=10pt,                          % Distance between numbers and code
  tabsize=2,                               % Tab size (2 spaces for indentation)
  showstringspaces=false,                  % Do not display spaces in strings
  breaklines=true,                         % Break long lines of code
  breakatwhitespace=false,                 % Allow breaking at arbitrary whitespace
  keywordstyle=\color{blue},               % Keywords in blue
  commentstyle=\color{green!50!black},     % Comments in dark green
  stringstyle=\color{red},                 % Strings in red
  basicstyle=\ttfamily\footnotesize,       % Code font and size
  framerule=2pt,                           % Thickness of the frame
}
\title{\Huge \textbf{Software Tools And Technology}} 
\date{}
\author{Group 32}

\begin{document}

\begin{figure}
    \centering
    \includegraphics[width=0.3\linewidth]{makaut logo.png}
\end{figure}

\maketitle
\pagenumbering{gobble}:

\begin{tikzpicture}
[remember picture, overlay] \draw[line width = 2pt] ($(current page.north west) + (0.5in, -0.5in)$) rectangle ($(current page.south east) + (-0.5in, 0.5in)$);
\end{tikzpicture}

\begin{center}

\LARGE\textbf{Lab Notebook}

\end{center}

\centering
\vspace{0.5cm}

\bfseries{\underline{Group members:}}

\begin{enumerate}
       \item Koushik Patra, BCA (Leader)
    \item Saniya Kayal, BCA
    \item Susmita Das, Bsc IT (AI)
    \item Aishi Barik, Bsc Forensic Science
    \item Sayan Biswas Bsc IT (AI)
\end{enumerate}

\vspace{2cm}
 \textbf{Instructor:} Dr.Ayan Ghosh \\
 \textbf{Course:} Software Tools And Technology

\newpage

\usetikzlibrary{calc}

\begin{tikzpicture}
[remember picture, overlay] \draw[line width = 2pt] ($(current page.north west) + (0.5in, -0.5in)$) rectangle ($(current page.south east) + (-0.5in, 0.5in)$);
\end{tikzpicture}

\centering

\begin{center}
\Large{\bfseries\underline{Lab Notebook Entries}}
\end{center}

\vspace{1.5cm}
\section{Lab Entry by Koushik Patra}

\vspace{0.7cm}
\subsection{Experiment}

\vspace{0.5cm}

\begin{table}[ht]
\centering
\begin{tabular}{|p{50pt}|p{200pt}|}
\hline
\textbf{Sl. No.} & \textbf{Assignments} \\ \hline
1. & Introduction to Github and Github desktop version installation \\ \hline
\end{tabular}
\end{table}

%\vspace{1.2cm}
\section{Lab Entry by Saniya Kayal}
\subsection{Experiment}

\vspace{0.5cm}
\begin{table}[ht]
\centering
\begin{tabular}{|p{50pt}|p{200pt}|}
\hline
\textbf{Sl. No.} & \textbf{Assignments} \\ \hline
1. & Converting Submit button to Chin Tapak Dum Dum \\ \hline
\end{tabular}
\end{table}

\section{Lab Entry by Susmita Das}
\subsection{Experiment}

\vspace{0.5cm}
\begin{table}[ht]
\centering
\begin{tabular}{|p{50pt}|p{200pt}|}
\hline
\textbf{Sl. No.} & \textbf{Assignments} \\ \hline
1. & Making calculator in C  \\ \hline
\end{tabular}
\end{table}

\newpage

\usetikzlibrary{calc}

\begin{tikzpicture}
[remember picture, overlay] \draw[line width = 2pt] ($(current page.north west) + (0.5in, -0.5in)$) rectangle ($(current page.south east) + (-0.5in, 0.5in)$);
\end{tikzpicture}
\section{Lab Entry by Aishi Barik}
\subsection{Experiment}

\vspace{0.5cm}
\begin{table}[ht]
\centering
\begin{tabular}{|p{50pt}|p{200pt}|}
\hline
\textbf{Sl. No.} & \textbf{Assignments} \\ \hline
1. & Creating latex repository on github \\ \hline
\end{tabular}
\end{table}

\vspace{2.5cm}
\section{Lab Entry by Sayan Biswas}
\subsection{Experiment}

\vspace{0.5cm}
\begin{table}[ht]
\centering
\begin{tabular}{|p{50pt}|p{200pt}|}
\hline
\textbf{Sl. No.} & \textbf{Assignments} \\ \hline
1. & Introduction to latex \\ \hline
\end{tabular}
\end{table}

\newpage
\usetikzlibrary{calc}

\begin{tikzpicture}
[remember picture, overlay] \draw[line width = 2pt] ($(current page.north west) + (0.5in, -0.5in)$) rectangle ($(current page.south east) + (-0.5in, 0.5in)$);
\end{tikzpicture}
\title{Lab Entry By\\[0.5cm] \large Koushik Patra\\[0.5cm]
    \begin{figure}
    \centering
    \includegraphics[width=0.5\linewidth]{WhatsApp Image 2024-09-15 at 22.55.20_cb23c58b.jpg} 
\end{figure}
\author{}
\date{}

\maketitle
\section*{1. Introduction to GitHub and GitHub Desktop Version Installation}

In this section, we will explore what GitHub is, its features, and how to install the GitHub Desktop version. GitHub is a platform that enables developers to collaborate, track, and manage code versions through Git version control. It provides a web-based interface for easier management of Git repositories.

\subsection*{1. Introduction to GitHub}
GitHub is a web-based platform that allows developers to host and review code, manage projects, and build software collaboratively. Git is a version control system that enables tracking of changes, and GitHub adds a graphical interface and additional features on top of Git.

\subsection*{2. GitHub Desktop Installation}
GitHub Desktop is a GUI client that simplifies the interaction with Git and GitHub. It allows users to handle repositories, create branches, and manage commits without using the command line.

\subsubsection*{Steps to Install GitHub Desktop:}
\begin{enumerate}
    \item Visit the GitHub Desktop official website at \\\texttt{https://desktop.github.com/}.
    \item Click on the \texttt{Download for your OS} button (Windows/Mac).
    \item Follow the installation instructions provided by the installer.
    \item Once installed, sign in with your GitHub credentials to start using GitHub Desktop.
\end{enumerate}






\newpage
\title{\centering \Large \textbf{4. Creating LaTeX Repository on GitHub} \\[0.5cm]}
\author{\centering \Large \textbf{Lab Entry By} \\[0.5cm] \large \textbf{Sayan Biswas}}
\begin{document}

\maketitle

\section{Introduction}
In this guide, we will walk through the steps to create a \LaTeX\ repository on GitHub. GitHub is a popular platform for version control and collaboration, allowing you to store your \LaTeX\ projects and share them with others.

\section{Pre-requisites}
Before proceeding, make sure you have the following:
\begin{itemize}
    \item A GitHub account (\url{https://github.com/}).
    \item \texttt{git} installed on your machine.
    \item Basic understanding of version control.
\end{itemize}

\section{Step 1: Creating a Repository}
To create a repository on GitHub:
\begin{enumerate}
    \item Log in to your GitHub account.
    \item Click on the \textbf{New} repository button in the top right corner.
    \item Give your repository a name, e.g., \texttt{my-latex-project}.
    \item Optionally, add a description.
    \item Choose whether you want the repository to be public or private.
    \item Click \textbf{Create repository}.
\end{enumerate}

\section{Step 2: Setting Up \LaTeX\ Project Locally}
Next, set up your \LaTeX\ project locally:
\begin{enumerate}
    \item Navigate to the folder where you want to store your project.
    \item Initialize a \texttt{git} repository:
    \begin{verbatim}
    git init
    \end{verbatim}
    \item Create your \LaTeX\ files, for example:
    \begin{verbatim}
    touch main.tex
    \end{verbatim}
\end{enumerate}

\section{Step 3: Connecting Local Repository to GitHub}
Now, connect your local repository to GitHub:
\begin{enumerate}
    \item In your terminal, run:
    \begin{verbatim}
    git remote add origin https://github.com/your-username/my-latex-project.git
    \end{verbatim}
    \item Add and commit your changes:
    \begin{verbatim}
    git add .
    git commit -m "Initial commit"
    \end{verbatim}
    \item Push your changes to GitHub:
    \begin{verbatim}
    git push -u origin master
    \end{verbatim}
\end{enumerate}

\section{Step 4: Managing Your \LaTeX\ Project}
Once the repository is created and set up, you can continue to work on your \LaTeX\ project by adding, committing, and pushing changes. GitHub also allows for collaborative work, so you can invite others to contribute to your project.

\subsection{Adding Files}
To add new files, such as additional \LaTeX\ chapters or images, you can:
\begin{verbatim}
git add newfile.tex
git commit -m "Added new chapter"
git push
\end{verbatim}

\subsection{Collaboration}
To collaborate with others, simply add them as collaborators in your GitHub repository settings, or share the repository link for them to fork.

\section{Conclusion}
In this document, we covered how to create a \LaTeX\ repository on GitHub and manage it using basic \texttt{git} commands. By hosting your \LaTeX\ projects on GitHub, you can leverage version control and collaboration features to enhance your workflow.

\end{document}
